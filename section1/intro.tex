\IEEEPARstart{L}{ocalization} in randomly deployed WSNs has been a focus of interest in the research community. Its characteristics, like ease of deployment, suppose important advantages for some type of applications (nodes can be air-dropped off airplanes~\cite{airDroppedVolvano}). Global Positioning Systems (GPS) had been used to locate each node in the network. Nevertheless, because of the tightly constrained power source equipped in these nodes (normally two AA batteries) reducing the number of GPS modules is a viable way to increase the network lifetime while decreasing the budget.

To spread the implementation of this type of networks, localization protocols try to take the most out of the extremely constrained resources available. Limited battery, constrained processing power, constrained form-factor and cost are some of the limitations faced by each node~\cite{AkyildizWSNs}.

Localization protocols are often divided into two categories, called range-based and range-free. The former makes use of ranging techniques like Received Signal Strength Indicator (RSSI) in order to make straight-line distance estimations between the not-located nodes (called \emph{unknown}) and a reference node (called \emph{Anchor}) which broadcasts its location information in a packet type called \emph{Beacon}. The latter category just performs position estimations based on the effective connections among nodes. 

In some cases, one category might be more suitable than the other. For example, applications requiring coarse accuracy and running for very long periods of time might only need the simplicity offered by some range-free localization protocols. On the other hand, high-accuracy-demanding applications ask for localization protocols able to comply with strict accuracy requirements which are often achieved by combining several ranging techniques.

Although there are numerous protocols, none has proved to outperform the others under all possible scenarios and conditions; in \cite{composability} the authors combine and coordinate the execution of different position estimation protocols (from here on referred to as \emph{composability} of localization protocols) in order to leverage the weaknesses of some with the strengths of others. Their proposal proved to be effective and capable of locating 100\% of the nodes in the deployment. Nevertheless, there are no specifics regarding the impact on battery consumption, estimation error and order of protocol execution.

%Nevertheless, in their solution the order of protocol execution has to be previously defined and lacks of consideration of its impact on battery consumption, error and localization time.

This work extends the contribution of~\cite{composability} by addressing these issues and proposes a distributed localization procedure for randomly deployed WSNs. This is achieved by having a clear understanding of the selected localization protocols' best-working conditions and network deployment considerations. A localization protocol is found suitable when, while complying with the deployment considerations, its best-working conditions are also reached.

A short literature review is presented in Section~\ref{literature} and in Section~\ref{locProc} the proposed localization procedure is described. Simulation results with two example localization protocols are shown in Section~\ref{simulation} and finally conclusions are drawn in Section~\ref{conclusions}.
%In order to improve the simulation's results, more accurate ranging techniques need to be used to leverage the high levels of errors provided by RSSI ranging technique. As for the range-free protocol used, its performance can be enhanced by incorporating ranging measurements in order to add constraints to the underlying optimization problem (as in~\cite{convexEstimation}). These measures will result in an overall reduction of the error.

%The presented localization procedure considers the \emph{unknown} nodes' environmental conditions in order to make a more intelligent protocol selection. Coupled with the deployment considerations, nodes avoid incurring in additional battery consumption related to simultaneous protocol execution (as it is the case in~\cite{composability}), given that the PME selects a single protocol each time.

%In the simulation, by correctly identifying the best-working environmental conditions of each protocol the localization procedure is able to locate more nodes than by their individual execution. Also, the localization procedure effectively locates 100\% of the nodes in the deployment at \emph{Anchor} densities around $20$\% in a free space model, while the individual execution of the tested range-based localization protocol achieves the same at around $40$\%.

%Apart from the specific enhancements suggested for each protocol implementation, upon localization a newly located node may broadcast its estimated location so other \emph{unknown} nodes may enrich their environmental conditions. This location information exchange by newly located nodes introduces new challenges related to listening time and error propagation \cite{alsindi2006error}. Given that with this new approach the \emph{unknown} nodes would not only listen to \emph{Anchors} but also to other located nodes, the overall energy consumption would be increased.

%Exchanging the environmental conditions of each node opens the door for more complex and centralized localization algorithms \cite{pal2010localization,alippi2006rssi}. This requires specifics on how the \emph{unknown} node should identify the conditions where a centralized localization protocol is better for the given deployment considerations, what are the favorable environmental conditions and how these can be matched into deployment considerations.

In this work a new and flexible approach to the localization problem in randomly-deployed WSNs is presented. It extends the proposal of~\cite{composability}, which considers the composability of localization protocols as a robust solution. 

The localization procedure incorporates flexibility on the selection of localization protocols by determining which is more capable of achieving predetermined deployment considerations under the environmental conditions surrounding each \emph{unknown} node. Furthermore, it is designed to admit several localization protocols, definitions of environmental conditions and deployment considerations; which makes it a good choice for random deployments, like~\cite{airDroppedVolvano}.

A set of evaluations were preformed with two well-know localization protocols, referred to as Lateration (range-based) and Bounding-Box (range-free). Results show that the localization procedure is able to locate more nodes than by their individual execution, suggesting a more intelligent and flexible localization scheme that considers the current state of the nodes before making decisions about its future state.

% Also, the localization procedure effectively locates 100\% of the nodes in the deployment at \emph{Anchor} densities around $20$\% in a free space model, while the individual execution of the tested range-based localization protocol achieves the same at around $40$\%.

In order to improve the current proposal, it is important to identify the environment metrics that correlate with the performance of each localization protocol to be used. Once understood, simple adjustments in the PME would enable it to comply with the deployment considerations in a more effective way. Moreover, the PME can be adapted to make a protocol selection based not only on its own, but also with the surrounding nodes' environmental conditions. This opens the door to more complex and centralized localization algorithms, like~\cite{pal2010localization}~and~\cite{alippi2006rssi}.